\documentclass{article}

\usepackage[OT4]{polski}
\usepackage[utf8]{inputenc}
\usepackage{indentfirst}
\usepackage[left=2cm,right=2cm,top=1.5cm,bottom=2cm,includeheadfoot,a4paper]{geometry}
\usepackage{tabularx}
\usepackage{graphicx}
\usepackage{marginnote}
\usepackage{tabularx}
\usepackage{multicol}
\usepackage{hyperref}

\geometry{a4paper}
\linespread{1.0}
\frenchspacing
\renewcommand{\arraystretch}{1.4}

\title{
	Projektowanie i programowanie obiektowe II\\
	\Huge{UML: diagramy zachowań}
}
\author{mgr inż. Krzysztof Rewak}
\date{\today}

\begin{document}
	\maketitle

	\section{Zunifikowany język modelowania}
	Należy odpowiedzieć na pytania:
	\begin{itemize}
		\item jakie figury oznaczają jakie czynności w diagramach czynności?
		\item jakie jest zastosowanie diagramów przypadków użycia?
		\item czy diagram stanów można zaprojektować dla każdego programu?
		\item do czego służą diagramy interakcji?
	\end{itemize}
	
	\section{Zadanie}
	Należy:
	\begin{itemize}
		\item pobrać ze swojego repozytorium jeden z programów (od laboratorium piątego i wzwyż) zaimplementowanych na zajęciach z Projektowania i programowania obiektowego I: \url{http://pwsz.rewak.pl/kursy/1};
		\item zastanowić się nad diagramem zachowań wybranego projektu; warto przejrzeć przykładowe diagramy:
		\begin{itemize}
			\item \url{https://www.uml-diagrams.org/activity-diagrams.html}
			\item \url{https://www.uml-diagrams.org/state-machine-diagrams.html}
			\item \url{https://www.uml-diagrams.org/use-case-diagrams.html}
			\item \url{https://www.uml-diagrams.org/sequence-diagrams.html}
		\end{itemize}
		\item za pomocą narzędzia \href{https://www.draw.io/}{draw.io} utworzyć diagram zachowań dla wybranego projektu;
		\item gotowy schemat (najlepiej w formie graficznej oraz XML) dołączyć do repozytorium Git; zalecane jest uporządkowanie zadań w odpowiadającym im katalogach
	\end{itemize}

\end{document}