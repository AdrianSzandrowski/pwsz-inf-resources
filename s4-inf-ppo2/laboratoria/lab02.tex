\documentclass{article}

\usepackage[OT4]{polski}
\usepackage[utf8]{inputenc}
\usepackage{indentfirst}
\usepackage[left=2cm,right=2cm,top=1.5cm,bottom=2cm,includeheadfoot,a4paper]{geometry}
\usepackage{tabularx}
\usepackage{graphicx}
\usepackage{marginnote}
\usepackage{tabularx}
\usepackage{multicol}
\usepackage{hyperref}

\geometry{a4paper}
\linespread{1.0}
\frenchspacing
\renewcommand{\arraystretch}{1.4}

\title{
	Projektowanie i programowanie obiektowe II\\
	\Huge{UML: diagramy struktur}
}
\author{mgr inż. Krzysztof Rewak}
\date{\today}

\begin{document}
	\maketitle

	\section{Zunifikowany język modelowania}
	Należy odpowiedzieć na pytania:
	\begin{itemize}
		\item jaka jest różnica między diagramem klas a diagramem obiektów?
		\item jaki kształt przyjmuje reprezentacja klasy?
		\item co oznaczają dodane do atrybutów modyfikatory \texttt{+}, \texttt{\#}, \texttt{-} i \texttt{\~}?
		\item czym są:
		\begin{itemize}
			\item zależności?
			\item agregacje?
			\item asocjacje?
			\item generalizacje?
			\item kompozycje?
		\end{itemize}
	\end{itemize}
	
	\section{Zadanie}
	Należy:
	\begin{itemize}
		\item pobrać ze swojego repozytorium jeden z programów (od laboratorium piątego i wzwyż) zaimplementowanych na zajęciach z Projektowania i programowania obiektowego I: \url{http://pwsz.rewak.pl/kursy/1};
		\item zastanowić się nad diagramem klas wybranego projektu; warto przejrzeć przykładowe diagramy na stronie \url{https://www.uml-diagrams.org/class-diagrams-examples.html}
		\item za pomocą narzędzia \href{https://www.draw.io/}{draw.io} lub innego utworzyć diagram klas dla wybranego projektu;
		\item gotowy schemat (najlepiej w formie graficznej oraz XML) dołączyć do repozytorium Git; zalecane jest uporządkowanie zadań w odpowiadającym im katalogach
	\end{itemize}
	
	\subsection{Praca z repozytorium}
	Aby nie generować nieporozumień i problemów:
	\begin{itemize}
		\item proszę o utworzenie nowego repozytorium i  nazwanie go \texttt{PPO2 - Imię Nazwisko};
		\item repozytorium powinno być prywatne (najlepiej służy do tego Bitbucket);
		\item należy udostępnić repozytorium prowadzącemu zajęcia (adres jest dostępny na stronie internetowej w zakładce \emph{Kontakt});
		\item zaleca się utworzenie osobnych katalogów dla kolejnych zadań.
	\end{itemize}

\end{document}