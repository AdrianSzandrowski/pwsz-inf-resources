\documentclass{article}

\usepackage[OT4]{polski}
\usepackage[utf8]{inputenc}
\usepackage{indentfirst}
\usepackage[left=2cm,right=2cm,top=1.5cm,bottom=2cm,includeheadfoot,a4paper]{geometry}
\usepackage{tabularx}
\usepackage{graphicx}
\usepackage{marginnote}
\usepackage{tabularx}
\usepackage{multicol}
\usepackage{hyperref}

\geometry{a4paper}
\linespread{1.0}
\frenchspacing
\renewcommand{\arraystretch}{1.4}

\title{
	Projektowanie i programowanie obiektowe II\\
	\Huge{Techniczna dokumentacja projektu}
}
\author{mgr inż. Krzysztof Rewak}
\date{\today}

\begin{document}
	\maketitle
	
	\section{Zadanie}
	Należy:
	\begin{itemize}
		\item pobrać ze swojego repozytorium jeden z programów (od laboratorium piątego i wzwyż) zaimplementowanych na zajęciach z Projektowania i programowania obiektowego I: \url{http://pwsz.rewak.pl/kursy/1};
		\item udokumentować zgodnie z zasadami języka kod wybranego programu; 
		\item za pomocą phpDocumentora, Doxygenu, pydoca lub dowolnego innego generatora dokumentacji wytworzyć dokumentację wybranego programu;
		\item gotową dokumentację dołączyć do repozytorium Git; zalecane jest uporządkowanie zadań w odpowiadającym im katalogach
	\end{itemize}

\end{document}