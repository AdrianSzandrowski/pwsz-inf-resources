\documentclass{article}

\usepackage[OT4]{polski}
\usepackage[utf8]{inputenc}
\usepackage{indentfirst}
\usepackage[left=2cm,right=2cm,top=1.5cm,bottom=2cm,includeheadfoot,a4paper]{geometry}
\usepackage{tabularx}
\usepackage{graphicx}
\usepackage{easylist}
\usepackage{marginnote}

\geometry{a4paper}
\linespread{1.0}
\frenchspacing
\renewcommand{\arraystretch}{1.4}

\title{
	Projekt zespołowy \\
	\Huge{Zajęcia wstępne}
}
\author{mgr inż. Krzysztof Rewak}
\date{\today}

\begin{document}
	\maketitle

	\section{Warunki zaliczenia}
	Zaliczenie zajęć projektowych kursu \textbf{Projekt zespołowy} odbywa się poprzez opracowanie projektu programistycznego i sprawozdania oraz prezentację pracy projektowej. Ocena końcowa $\Omega$ będzie wyliczana w następujący sposób:
	
	\begin{equation} \Omega = 0.6 k_1 + 0.2 k_2 + 0.2 k_3 \label{omega} \end{equation}
	
	gdzie kolejne $k_n$ powinny być rozumiane następująco:
	
	\begin{itemize}
		\item $k_1$ - ocena za pracę projektową;
		\item $k_2$ - ocena za sprawozdanie z pracy projektowej;
		\item $k_3$ - ocena za prezentację pracy projektowej.
	\end{itemize}
	
	\subsection{Praca projektowa}
	Należy zaprojektować i zaimplementować projekt programistyczny wedle własnego pomysłu lub wybrany spośród proponowanych tematów w punkcie 2. tego dokumentu. Warunki zaliczenia kształtują się następująco:
	
	\begin{itemize}
		\item projekt jest wykonywany w grupach, dwu- lub trzysobowych;
		\item temat projektu musi zostać zatwierdzony (również eksternistycznie) przez prowadzącego zajęcia najpóźniej do terminu trzeciech zajęć;
		\item projekt może być wykonany w jednej z trzech form: jako aplikacja desktopowa, webowa lub mobilna;
		\item projekt może być wykonany w dowolnym języku lub językach programowania, z wykorzystaniem dowolnych bibliotek, komponentów lub wtyczek;
		\item aplikacja projektowa powinna być wyposażona w graficzny interfejs użytkownika.
	\end{itemize}
	
	\subsection{Sprawozdanie z pracy projektowej}
	Należy opracować sprawozdanie z pracy projektowej. Warunki zaliczenia kształtują się następująco:
	
	\begin{itemize}
		\item zgodnie z ogólnie przyjętami standardami naukowymi, sprawozdanie powinno być skonstruowane w technologii \LaTeX;
		\item sprawozdanie musi być dostarczone najpóźniej w dniu oddania całego projektu;
		\item sprawozdanie powinno zawierać (jako rozdziały lub sekcje):
		\begin{itemize}
			\item opis funkcjonalny systemu;
			\item streszczenie opisu technicznego;
			\item instrukcję uruchomienia programu;
			\item wnioski z projektu.
		\end{itemize}
	\end{itemize}
	
	\subsection{Prezentacja pracy projektowej}
	Należy przestawić całym zespołem efekty swojej pracy. Prezentacja może odbyć się w formie przedstawienia działającej aplikacji, krótkiej prezentacji multimedialnej lub innej formy. Ostatnim etapem prezentacji będą pytania od prowadzącego zajęcia.
	
	\section{Przykładowe tematy prac projektowych}
	\begin{itemize}
		\item system rezerwacji biletów w kinie;
		\item system do tworzenia i uzupełniania ankiet;
		\item system zarządzania hotelem;
		\item system zarządzania domowymi funduszami;
		\item system do zarządzania treścią online na stronie www;
		\item system do zapisywania i oceniania przeczytanych książek;
		\item system do zarządzania magazynem;
		\item system informacyjny wykorzystujący zewnętrzne API;
		\item system do harmonogramowania pracy w zakładzie produkcyjnym.
	\end{itemize}

\begin{thebibliography}{9}	
	\bibitem{martin} 
	Robert C. Martin.
	\textit{Czysty kod. Podręcznik dobrego programisty}. 
	Wydawnictwo Helion, 2010.
\end{thebibliography}



\end{document}