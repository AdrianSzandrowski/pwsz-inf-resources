\documentclass{article}

\usepackage[OT4]{polski}
\usepackage[utf8]{inputenc}
\usepackage{indentfirst}
\usepackage[left=2cm,right=2cm,top=1.5cm,bottom=2cm,includeheadfoot,a4paper]{geometry}
\usepackage{tabularx}
\usepackage{graphicx}
\usepackage{marginnote}
\usepackage{tabularx}
\usepackage{multicol}

\geometry{a4paper}
\linespread{1.0}
\frenchspacing
\renewcommand{\arraystretch}{1.4}

\title{
	Projektowanie i programowanie systemów internetowych I\\
	\Huge{Zasady zaliczenia zajęć projektowych}
}
\author{mgr inż. Krzysztof Rewak}
\date{\today}

\begin{document}
	\maketitle

	\section{Warunki zaliczenia}
	Zaliczenie zajęć projektowych kursu \textbf{Projektowanie i programowanie systemów internetowych} odbywa się poprzez opracowanie projektu programistycznego i sprawozdania oraz prezentację pracy projektowej. Ocena końcowa $\Omega$ będzie wyliczana w następujący sposób:
	
	\begin{equation} \label{eq:someequation}
		\Omega = 0.5k_1 + 0.3k_2 + 0.2k_3
	\end{equation}
	
	gdzie kolejne $k_n$ powinny być rozumiane następująco:
	
	\begin{itemize}
		\item $k_1$ - ocena za pracę projektową;
		\item $k_2$ - ocena za sprawozdanie z pracy projektowej;
		\item $k_3$ - ocena za prezentację pracy projektowej.
	\end{itemize}
	
	Oceniany jest zespół oddający pracę, zatem wszyscy członkowie zespołu otrzymują identyczną ocenę końcową. Należy to rozumieć dwojako:
	\begin{itemize}
		\item jeżeli tylko jedna osoba w zespole potrafi odpowiedzieć na pytanie prowadzącego zajęcia, zespół zostaje oceniony pozytywnie;
		\item jeżeli żadna osoba w zespole nie potrafi odpowiedzieć na pytanie prowadzącego zajęcia, zespół zostaje oceniony negatywnie.
	\end{itemize}
	
	Oddanie pracy projektowej wymaga obecności całego zespołu, a warunkiem koniecznym jest $k_n > 2.0$
	
	\subsection{Praca projektowa}
	Należy zaprojektować, zaimplementować i wdrożyć system internetowy wedle własnego pomysłu lub na podstawie wybranego spośród proponowanych tematów w punkcie 2. tego dokumentu. Warunki zaliczenia kształtują się następująco:
	
	\begin{itemize}
		\item projekt jest wykonywany w parach:
		\begin{itemize}
			\item w przypadku niedobrania pary, parę wyznacza prowadzący zajęcia;
			\item w przypadku nieparzystej liczby studentów, prowadzący może utworzyć zespół trzyosobowy;
		\end{itemize}
		\item unikalny temat projektu musi zostać zatwierdzony przez prowadzącego zajęcia najpóźniej do terminu trzeciech zajęć; w przeciwnym razie grupa otrzymuje:
		\begin{itemize}
			\item wybrany przez prowadzącego temat bez możliwości zmiany tematu;
			\item modyfikator $-0.5$ do oceny końcowej za projekt za niedotrzymanie terminu;
		\end{itemize}
		\item podział prac w zespole powinien zostać zdefiniowany na początku pracy i zgłoszony wraz z tematem projektu;
		\item projekt może być wykonany w dowolnym języku lub językach programowania, z wykorzystaniem dowolnych bibliotek, komponentów lub wtyczek;
	\end{itemize}
	
	\newpage
	
	Praca projektowa musi zawierać przynajmniej $n$ z podanych poniżej dwudziestu zagadnień kwalifikacyjnych. Za każdą kwalifikację zostanie wystawiona ocena cząstkowa, a z nich zostanie wyliczona ocena $k_1$, ale niewyższa niż $f(n)$.
	
	\begin{center}
		\begin{tabularx}{.5\textwidth}{X|X}
		\textbf{n} & \textbf{ f(n) } \\ \hline
		$ \geq 16 $ &  5.0 \\ \hline
		$ 14-15 $ &  4.5 \\ \hline
		$ 12-13 $ &  4.0 \\ \hline
		$ 10-11 $ &  3.5 \\ \hline
		$ 8-9 $ &  3.0 \\ \hline
		$ \le 8 $ &  2.0 \\
		\end{tabularx}
	\end{center}
	
	\renewcommand{\arraystretch}{1.9}
	
	\begin{tabularx}{.95\textwidth}{l|X|l}
		\textbf{\#} & \textbf{Zagadnienie} & \textbf{ Ocena \ \ \ } \\ \hline
		1 & \textbf{HTML5}, szkielet aplikacji webowej \\ \hline
		2 & \textbf{CSS3}, ostylowanie aplikacji webowej \\ \hline
		3 & \textbf{formularze}, wykorzystanie formularza do przesyłu danych do aplikacji \\ \hline
		4 & \textbf{baza danych}, połączenie z zewnętrzną bazą danych \\ \hline
		5 & \textbf{router}, zarządzenie ścieżkami dostępu aplikacji \\ \hline
		6 & \textbf{uwierzytelnianie}, umożliwienie rejestracji i logowania użytkowników \\ \hline
		7 & \textbf{MVC}, wykorzystanie wzorca model-widok-kontroler \\ \hline
		8 & \textbf{CRUD}, operacje dodawania, edycji i usuwania rekordów z bazy danych \\ \hline
		9 & \textbf{ORM}, system mapowania relacyjno-obiektowego \\ \hline
		10 & \textbf{wystawienie API}, wystawienie udokumentowanych adresów URL z informacjami \\ \hline
		11 & \textbf{konsumowanie API}, wykorzystanie zewnętrznego API w systemie \\ \hline
		12 & \textbf{AJAX}, wykorzystanie asynchronicznych zapytań ze strony frontendu\\ \hline
		13 & \textbf{mail}, wysyłanie emaili z systemu \\ \hline
		14 & \textbf{lokalizacja}, umożliwienie wielojęzyczności systemu \\ \hline
		15 & \textbf{RWD}, poprawne wyświetlanie frontendu na różnych urządzeniach \\ \hline
		16 & \textbf{logger}, logowanie akcji w aplikacji \\ \hline
		17 & \textbf{cache}, np. wyszukiwarka wykorzystująca system pamięci podręcznej \\ \hline
		18 & \textbf{system zarządzania zależnościami} wykorzystany przy projekcie\\ \hline
		19 & \textbf{automatyzacja}, na przykład wykorzystanie Sass albo minifikowanie JS \\ \hline
		20 & \textbf{SEO}, wprowadzenie optymalizacji dla wyszukiwarek \\
	\end{tabularx}
	
	\subsection{Sprawozdanie z pracy projektowej}
	Należy opracować sprawozdanie z pracy projektowej. Warunki zaliczenia kształują się następująco:
	\begin{itemize}
		\item zgodnie z ogólnoprzyjętymi standardami naukowymi, sprawozdanie jest skonstruowane w technologii \LaTeX \ i oddane w formacie \texttt{.pdf};
		\item sprawozdanie jest dostarczone najpóźniej w dniu oddania całego projektu w formie elektronicznej lub papierowej;
		\item sprawozdanie zawiera jako rozdziały lub sekcje:
		\begin{itemize}
			\item opis funkcjonalny systemu
			\item wyszczególnione wdrożone kwalifikacje z punktu 1.1
			\item streszczenie opisu technologicznego
			\item instrukcję lokalnego i zdalnego uruchomienia systemu
			\item wnioski projektowe.
		\end{itemize}
	\end{itemize}	
	
	\subsection{Prezentacja pracy projektowej}
	Należy przestawić całym zespołem efekty swojej pracy. Prezentacja musi odbyć się w formie przedstawienia działającej aplikacji. Dopuszczalne są dodatkowe formy takie jak krótka prezentacja multimedialna.
	
	\section{Przykładowe tematy prac projektowych}
	\begin{multicols}{2}
		\begin{itemize}
			\item blog
			\item system z ogłoszeniami
			\item sklep internetowy
			\item kalendarz urodzin
			\item portal informacyjny
			\item system do tworzenia dokumentacji projektów informatycznych
			\item forum internetowe
			\item e-dziennik
			\item czytnik RSS
			\item klony lub adaptacje popularnych systemów:
			\begin{itemize}
				\item Endomondo
				\item GoodReads albo lubimyczytac.pl
				\item Instagram
				\item Tinder
			\end{itemize}
		\end{itemize}
	\end{multicols}
	
	\section{Terminarz}
	\begin{multicols}{2}
		\begin{itemize}
			\item \textbf{1.: Zajęcia wstępne}
			\item 2.: Praca zespołowa
			\item \textbf{3.: Termin deklaracji tematu}
			\item 4.: Praca zespołowa
			\item 5.: Praca zespołowa
			\item 6.: \textbf{Termin dostarczenia adresu i udzielenia dostępów do repozytorium projektu}
			\item 7.: Praca zespołowa
			\item 8.: \textbf{Połowa semestru}
			\item 9.: Praca zespołowa
			\item 10.: Praca zespołowa
			\item \textbf{11.: Sugerowane kończenie projektów}
			\item 12.: Praca zespołowa
			\item 13.: Praca zespołowa
			\item 14.: Praca zespołowa
			\item \textbf{15.: Ostateczny termin oddania projektu}
		\end{itemize}
	\end{multicols}

\end{document}