\documentclass{article}

\usepackage[OT4]{polski}
\usepackage[utf8]{inputenc}
\usepackage{indentfirst}
\usepackage[left=2cm,right=2cm,top=1.5cm,bottom=2cm,includeheadfoot,a4paper]{geometry}
\usepackage{tabularx}
\usepackage{graphicx}
\usepackage{marginnote}
\usepackage{hyperref}
\usepackage{listings}

\geometry{a4paper}
\linespread{1.1}
\frenchspacing
\renewcommand{\arraystretch}{1.2}

\title{
	Projektowanie i programowanie obiektowe \\
	\Huge{Cechy}
}
\author{mgr inż. Krzysztof Rewak}
\date{\today}

\begin{document}
	\maketitle
	
	\section{Cechy w programowaniu obiektowym}
	Należy uruchomić w dowolnym środowisku załączony program z katalogu \texttt{lab10}. Programy w obu językach wykonują dokładnie to samo zadanie.
	
	\subsection{Java}
	W katalogu \texttt{lab10/java} znajduje się kod źródłowy aplikacji podzielonej na pakiety. Należy zbudować, skompilować i uruchomić projekt w dowolny sposób. Zalecane jest środowisko IntelliJ IDEA (dostępne dla studentów za darmo).
	
	\subsection{PHP}
	W katalogu \texttt{lab10/php} znajduje się kod źródłowy aplikacji wymagającej autoloadera. Należy uruchomić komendę \texttt{composer install}, aby utworzyć katalog \texttt{vendor} z wymaganymi plikami. Aplikację uruchamia się poprzez polecenie \texttt{php index.php} w głównym katalogu. Zalecane jest środowisko PHPStorm (dostępne dla studentów za darmo).
	
	\subsection{Pytania do zadania}
	Należy odpowiedzieć na następujące pytania dotyczące klas abstrakcyjnych:
	\begin{itemize}
		\item czym jest cecha?
		\item brak jakiej funkcjonalności mogą zastąpić cechy?
		\item jaka jest różnica między cechą w PHP i \emph{cechą} w Javie?
	\end{itemize}

	\section{Zadanie programistyczne}
	Należy rozszerzyć program:
	
	\begin{itemize}
		\item program powinien umożliwić dodanie przynajmniej pięciu typów notatek;
		\item opcje menu powinny zostać wyabstrahowane;
		\item slugi powinny być unikalne, a więc powtórki powinny dodawać numer: \texttt{test}, \texttt{test-2}, \texttt{test-3}...
		\item slug powinny być wyświetlane alfabetycznie.
	\end{itemize}
	
	Plik z programem należy dołączyć do repozytorium Git. Zalecane jest uporządkowanie zadań w odpowiadającym im katalogach.

\end{document}