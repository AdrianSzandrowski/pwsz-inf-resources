\documentclass{article}

\usepackage[OT4]{polski}
\usepackage[utf8]{inputenc}
\usepackage{indentfirst}
\usepackage[left=2cm,right=2cm,top=1.5cm,bottom=2cm,includeheadfoot,a4paper]{geometry}
\usepackage{tabularx}
\usepackage{graphicx}
\usepackage{marginnote}
\usepackage{hyperref}
\usepackage{listings}

\geometry{a4paper}
\linespread{1.1}
\frenchspacing
\renewcommand{\arraystretch}{1.2}

\title{
	Projektowanie i programowanie obiektowe \\
	\Huge{Przeciążanie operatorów}
}
\author{mgr inż. Krzysztof Rewak}
\date{\today}

\begin{document}
	\maketitle

	\section{Przeciążanie operatorów}
	Należy pobrać, dokładnie przeanalizować, skompilować i uruchomić załączony projekt \texttt{lab08}, a następnie odpowiedzieć na pytania:
	
	\begin{itemize}
		\item jaki operator został przeciążony dla klasy \texttt{Library}?
		\item czym są pliki nagłówkowe \texttt{.h} i w jakim celu się je dodaje do projektu?
		\item w jakiej strukturze danych przetrzymywane są książki w bibliotece?
		\item co zwraca operacja dodania książki do biblioteki?
		\item co się stanie po operacji \texttt{Library("Biblioteka nr 1") + Library("Biblioteka nr 2")}?
		\item co robi linijka \texttt{setlocale (LC\_ALL, "");}?
		\item do czego sensownie można używać przeciążenia operatorów?
	\end{itemize}

	\section{Zadanie programistyczne}
	Należy utworzyć program realizujący podobną, ale rozszerzoną funkcjonalność. Program może być wykonany w dowolnej technologii pozwalającej na przeciążanie operatorów. Program powinien realizować następujące funkcjonalności:
	
	\begin{itemize}
		\item tworzenie biblioteki, która będzie miała nazwę oraz działy tematyczne;
		\item dodawanie nazywalnych działów do biblioteki za pomocą operacji dodawania \texttt{library += section};
		\item dodawanie książek (tytuł, autor) do działu za pomocą operacji dodawania \texttt{section += book}; zalecane jest wygenerowanie $n> 10$ książek dla każdego działu;
		\item łączenie działów \texttt{section3 = section2 + section1} łączące książki oraz \texttt{section2 += section1} dodające książki z pierwszego działu do drugiego;
		\item usuwanie konkretnego działu z biblioteki po obiekcie działu (\texttt{library -= section}) oraz nazwie (\texttt{library -= "fantastyka"});
		\item wypisywanie wszystkich książek danej biblioteki.
	\end{itemize}
	
	Plik z programem należy dołączyć do repozytorium Git. Zalecane jest uporządkowanie zadań w odpowiadającym im katalogach.

\end{document}