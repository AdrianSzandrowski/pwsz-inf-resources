\documentclass{article}

\usepackage[OT4]{polski}
\usepackage[utf8]{inputenc}
\usepackage{indentfirst}
\usepackage[left=2cm,right=2cm,top=1.5cm,bottom=2cm,includeheadfoot,a4paper]{geometry}
\usepackage{tabularx}
\usepackage{graphicx}
\usepackage{marginnote}
\usepackage{hyperref}
\usepackage{listings}

\geometry{a4paper}
\linespread{1.1}
\frenchspacing
\renewcommand{\arraystretch}{1.2}

\title{
	Projektowanie i programowanie obiektowe \\
	\Huge{Wielokrotne dziedziczenie klas}
}
\author{mgr inż. Krzysztof Rewak}
\date{\today}

\begin{document}
	\maketitle

	\section{Wielokrotne dziedziczenie klas w programowaniu obiektowym}
	Należy zbudować i uruchomić załączony plik \texttt{lab06.py} z dodatkowym plikiem \texttt{force\_users.py}. Proponowane środowisko programistyczne to PyCharm w wersji Community Edition z Pythonem w wersji 2.7. Należy przeanalizować kod, a następnie odpowiedzieć na pytania:
	\begin{itemize}
		\item jak wygląda hierarchia dziedziczenia w przykładowym programie?
		\item czym się różni dziedziczenie w Pythonie od dziedziczenia w PHP?
		\item czym jest funkcja \texttt{\_\_init\_\_()}?
		\item ilu różnych form ataku może użyć lord Sithów?
		\item czym jest funkcja \texttt{super()}?
		\item jakie mogłoby być uzasadnione wykorzystanie przykładowego programu?
	\end{itemize}

	\section{Zadanie programistyczne}
	Należy rozszerzyć program z pliku \texttt{lab06.py} o następujące funkcjonalności:
	\begin{itemize}
		\item program pozwala wygenerować dowolną liczbę uczestników; generowani uczestnicy mogą mieć losowe nazwy lub po prostu numery identyfikacyjne; 
		\item należy dodać element losowości do siły ataku; można założyć poziom szumu na poziomie $\pm$ 10\%;
		\item należy dodać możliwość bronienia się przed atakiem; proponowane rozwiązanie to utworzenie dodatkowej klasy \texttt{Defence} z wartością wyrażoną w procentach oznaczającą możliwość wyjścia bezpiecznie z sytuacji; przy ataku uczestnika A na uczestnika B sprawdzana byłaby możliwość obrony uczestnika B i na jej podstawie atak by się powiódł, częściowo powiódł lub chybił;
		\item należy stworzyć system zapamiętywania w jakiej kolejności który uczestnik ,,odpadł'' z zawodów; chronologiczna lista powinna wyświetlić się przy zakończeniu programu.
	\end{itemize}
	
	Plik z programem należy dołączyć do repozytorium Git. Zalecane jest uporządkowanie zadań w odpowiadającym im katalogach.

\end{document}