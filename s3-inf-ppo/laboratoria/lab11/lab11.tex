\documentclass{article}

\usepackage[OT4]{polski}
\usepackage[utf8]{inputenc}
\usepackage{indentfirst}
\usepackage[left=2cm,right=2cm,top=1.5cm,bottom=2cm,includeheadfoot,a4paper]{geometry}
\usepackage{tabularx}
\usepackage{graphicx}
\usepackage{marginnote}
\usepackage{hyperref}
\usepackage{listings}

\geometry{a4paper}
\linespread{1.1}
\frenchspacing
\renewcommand{\arraystretch}{1.2}

\title{
	Projektowanie i programowanie obiektowe \\
	\Huge{Refleksje}
}
\author{mgr inż. Krzysztof Rewak}
\date{\today}

\begin{document}
	\maketitle
	
	\section{Refleksje w programowaniu obiektowym}
	Należy uruchomić w dowolnym środowisku załączony program z katalogu \texttt{lab11}. Programy w obu językach wykonują dokładnie to samo zadanie.
	
	\subsection{Java}
	W katalogu \texttt{lab11/java} znajduje się kod źródłowy aplikacji. Należy zbudować, skompilować i uruchomić projekt w dowolny sposób. Zalecane jest środowisko IntelliJ IDEA (dostępne dla studentów za darmo).
	
	\subsection{PHP}
	W katalogu \texttt{lab10/php} znajduje się kod źródłowy aplikacji wymagającej autoloadera. Należy uruchomić komendę \texttt{composer install}, aby utworzyć katalog \texttt{vendor} z wymaganymi plikami. Aplikację uruchamia się poprzez polecenie \texttt{php index.php} w głównym katalogu. Zalecane jest środowisko PHPStorm (dostępne dla studentów za darmo).
	
	\subsection{Pytania do zadania}
	Należy odpowiedzieć na następujące pytania dotyczące klas abstrakcyjnych:
	\begin{itemize}
		\item czym jest refleksja?
		\item czy metoda wywołana przez refleksje musi być publiczna?
		\item czy można refleksją wywołać metodę z innej klasy?
		\item czy można refleksją wywołać coś więcej niż metodę po nazwie?
		\item czy wykorzystywanie refleksji to dobra praktyka programistyczna?
		\item jakie zagrożenia niesie ze sobą wykorzystywanie refleksji?
		\item do czego można wykorzystać mechanizm refleksji?
	\end{itemize}

	\section{Zadanie programistyczne}
	Należy zmodyfikować program, tak aby umożliwić zarządzanie prostym blogiem:
	
	\begin{itemize}
		\item metoda \texttt{create} powinna utworzyć nowy (autoinkrementowany) post na blogu;
		\item metoda \texttt{list} powinna wygenerować listę postów;
		\item metoda \texttt{show} powinna wyświetlić dany post;
		\item metoda \texttt{random} powinna wyświetlić losowy post;
		\item metoda \texttt{delete} powinna usunąć (wszystkie posty? ostatni post? wybrany post?);
		\item metoda \texttt{quit} powinna wyjść z programu;
		\item blog powinien mieć możliwość zaimportowania repozytorium postów.
	\end{itemize}
	
	Plik z programem należy dołączyć do repozytorium Git. Zalecane jest uporządkowanie zadań w odpowiadającym im katalogach.

\end{document}