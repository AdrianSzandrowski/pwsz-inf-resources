\documentclass{article}

\usepackage[OT4]{polski}
\usepackage[utf8]{inputenc}
\usepackage{indentfirst}
\usepackage[left=2cm,right=2cm,top=1.5cm,bottom=2cm,includeheadfoot,a4paper]{geometry}
\usepackage{tabularx}
\usepackage{graphicx}
\usepackage{marginnote}
\usepackage{tabularx}
\usepackage{multicol}

\geometry{a4paper}
\linespread{1.0}
\frenchspacing
\renewcommand{\arraystretch}{1.4}

\title{
	Projektowanie i programowanie systemów internetowych II\\
	\Huge{Zasady zaliczenia zajęć projektowych}
}
\author{mgr inż. Krzysztof Rewak}
\date{\today}

\begin{document}
	\maketitle

	\section{Warunki zaliczenia}
	Zaliczenie zajęć projektowych kursu \textbf{Projektowanie i programowanie systemów internetowych II} odbywa się poprzez opracowanie projektu programistycznego i sprawozdania oraz prezentację pracy projektowej. Ocena końcowa dla studenta $\Omega$ będzie wyliczana w następujący sposób:
	
	\begin{equation} \label{eq:koncowa}
		\Omega = 0.5k_s + 0.5k_p
	\end{equation}
	
	gdzie
	
	\begin{equation} \label{eq:koncowa}
	 k_p = 0.5k_1 + 0.3k_2 + 0.2k_3
	\end{equation}
	
	gdzie kolejne $k_n$ powinny być rozumiane następująco:
	
	\begin{itemize}
		\item $k_s$ - ocena indywidualna za wkład do projektu;
		\item $k_1$ - ocena zespołu za pracę projektową;
		\item $k_2$ - ocena zespołu za sprawozdanie z pracy projektowej;
		\item $k_3$ - ocena zespołu za prezentację pracy projektowej.
	\end{itemize}
	
	Oddanie pracy projektowej wymaga obecności całego zespołu, a warunkiem koniecznym jest $k_n > 2.0$  Ze względu na projektowy charakter zajęć, niedostateczna ocena wystawiona pod koniec semestru nie może zostać poprawiona w terminie dodatkowym.
	
	\subsection{Praca projektowa}
	Należy zaprojektować, zaimplementować i wdrożyć system internetowy zgodnie z warunkami zdefiniowanymi poniżej:
	
	\begin{itemize}
		\item projekt jest wykonywany w grupach maksymalnie pięcioosobowych, dobranych na pierwszych zajęciach; każda grupa powinna wybrać lidera, który przez cały semestr będzie kontaktował się z prowadzącym projekt w sprawach związanych z zadaniem;
		\item każdy student powinien wybrać swoją rolę w grupie spośród dostępnych: programista backendu, programista frontendu, tester, dev-ops lub project manager; domyślnie lider zostaje project managerem, jednakże możliwe są bardziej skomplikowane podziały (np. pół-frontend/pół-PM); w jednej grupie role mogą się powtarzać, ale w ramach wymagań technicznych;
		\item każda grupa na pierwszych zajęciach wybierze jeden z tematów przedstawionych przez prowadzącego; każdy temat zostanie pokrótce wyjaśniony, a szczegóły uzyskają project managerowie z każdej grupy w drugiej części pierwszych zajęć; każdy projekt będzie miał własną sobie listę wymagań i oczekiwanych funkcjonalności;
		\item do terminu trzecich zajęć obowiązkowe jest nadanie prowadzącemu dostępów do repozytorium projektu;
		\item podczas semestru odbędą się maksymalnie trzy spotkania z liderami grup (terminy są wyszczególnione w punkcie 2. tego dokumentu); na pierwszym grupa powinna dostarczyć specyfikację projektu oraz przedstawić podział prac, na nastepnych - przedstawić podstępy nad pracami; zaleca się zorganizowanie pracy poprzez wykorzystanie odpowiednich narzędzi;
		\item projekt może być wykonany w dowolnym języku lub językach programowania, z wykorzystaniem dowolnych bibliotek, komponentów lub wtyczek, ale jako aplikacja webowa;
		\item projekt powinien zostać wdrożony w dowolny sposób na dowolnym serwerze internetowym.
	\end{itemize}
	
	\subsection{Sprawozdanie z pracy projektowej}
	Należy opracować sprawozdanie z pracy projektowej. Warunki zaliczenia kształują się następująco:
	\begin{itemize}
		\item zgodnie z ogólnoprzyjętymi standardami akademickimi i naukowymi, sprawozdanie powinno zostać skonstruowane w technologii \LaTeX \ i oddane w formacie \texttt{.pdf};
		\item sprawozdanie powinno zostać dostarczone najpóźniej w dniu oddania całego projektu w formie elektronicznej;
		\item sprawozdanie powinno zawierać (jako rozdziały lub sekcje):
		\begin{itemize}
			\item opis funkcjonalny systemu,
			\item streszczenie opisu technologicznego,
			\item instrukcję lokalnego i zdalnego uruchomienia systemu,
			\item instrukcję uruchomienia testów oraz opis testowanych funkcjonalności,
			\item link do dokumentacji projektu,
			\item wnioski projektowe.
		\end{itemize}
	\end{itemize}	
	
	\subsection{Prezentacja pracy projektowej}
	Należy zaprezentować całym zespołem efekty swojej pracy w formie przedstawienia działającej aplikacji. Dopuszczalne są dodatkowe formy takie jak krótka prezentacja multimedialna.
	
	\section{Terminarz}
	\begin{itemize}
		\item 1./1 X: Zajęcia wstępne, wybór grup i tematu
		\item 2./8 X: Praca zespołowa
		\item 3./15 X: Praca zespołowa, \textbf{ostateczny termin udzielenia dostępów do repozytorium projektu}
		\item 4./22 X: Praca zespołowa, spotkanie z liderami grup
		\item 5/29 XI.: Praca zespołowa,
		\item 6./5 XI: Praca zespołowa
		\item 7./12 XI: Praca zespołowa, spotkanie z liderami grup
		\item 8./19 XI: Praca zespołowa, połowa semestru
		\item 9./26 XI: Praca zespołowa
		\item 10./3 XII: Praca zespołowa
		\item 11./10 XII: Praca zespołowa, spotkanie z liderami grup, sugerowane kończenie projektów
		\item 12./17 XII: Praca zespołowa, oddawanie projektu
		\item (przerwa bożonarodzeniowa)
		\item 13./7 I: Praca zespołowa, oddawanie projektu
		\item 14./14 I: Praca zespołowa, oddawanie projektu
		\item 15./21 I: \textbf{Ostateczny termin oddania projektu}
	\end{itemize}

\end{document}